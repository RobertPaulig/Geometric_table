% name4.tex — Том II: развитие модели QSG v6.x+

\section*{Том II. Развитие модели QSG: версии v6.x и далее}

Настоящий текст продолжает изложение, начатое в документе
\emph{name3.tex}, где фиксированы онтология спектральной среды,
геометрическая таблица, версия QSG v5.0 и введён минимальный
Spectral Lab v1. Здесь собираются рабочие записки и расширения,
относящиеся к будущим версиям модели (условно QSG v6.x и далее),
включая углублённый Spectral Lab, генераторы с циклами, d-блок,
ядерный мост и дальнейшее развитие спектральной гравитации.

\subsection{Обзор направления QSG v6.x}

В качестве отправной точки принимается следующая конфигурация:

\begin{itemize}
  \item \textbf{Зелёная зона QSG v5.0:} геометрический атом H--Kr,
        D/A-индексы и роли, деревьевой Grower, функционал сложности
        (v1 и FDM) с константами $\alpha_{\mathrm{tree}}$, а также
        мягкость $s(Z)$ для ростовой динамики.
  \item \textbf{Spectral Lab v1 (QSG v6.0 в зачатке):} одномерный
        оператор $\hat{H}$ на решётке, спектр, DOS/LDOS и игрушечный
        функционал $F_{\mathrm{levels}}$ с FDM-прокси.
  \item \textbf{Жёлтая зона:} континуальный оператор $\hat{H}[\theta]$,
        честный DOS/LDOS, crossing-слой сложности, полноценный d-блок,
        строгий мост geom $\leftrightarrow F_{\mathrm{nuc}}$.
  \item \textbf{Красная зона:} спектральная гравитация и космология,
        многоуровневая солитонная онтология и полная интеграция всех
        слоёв в единую спектральную теорию.
\end{itemize}

Задача тома II --- пошагово продвигать модель от текущего состояния
к более полному спектральному описанию, при этом каждый шаг должен
быть \emph{привязан к коду и численным экспериментам}, а не оставаться
на уровне чисто словесных деклараций.

\subsection{Spectral Lab v1 как R\&D-площадка}

Цель Spectral Lab v1 --- служить минимальной лабораторией для проверки
идей о спектральном операторе $\hat{H}$ и функционале уровней
$F_{\mathrm{levels}}$, а также для отработки связи между спектральной
картиной и FDM-приближениями.

В \emph{name3.tex} описан базовый одномерный оператор
\[
  H = -\frac{1}{2m}\,\frac{\mathrm{d}^2}{\mathrm{d}x^2} + V(x)
\]
на равномерной решётке, реализованный в модуле
\texttt{core/spectral\_lab\_1d.py}, и две версии функционала уровней:
спектральную $F_{\mathrm{levels}}^{(\mathrm{spec})}$ и FDM-прокси
$F_{\mathrm{levels}}^{(\mathrm{FDM})}$ (модуль
\texttt{core/f\_levels\_1d.py}). Здесь мы фиксируем план экспериментов,
которые должны сделать Spectral Lab содержательным шагом в сторону
настоящего $\hat{H}[\theta]$.

\subsubsection{Эксперимент SL-1: разрешение решётки и согласие spec/FDM}

Первая задача Spectral Lab v1 --- количественно проверить, насколько
FDM-прокси для $F_{\mathrm{levels}}$ приближают спектральные значения
при разумных параметрах. Для этого вводится эксперимент SL-1.

В реализации QSG v6.0 эксперимент SL-1 оформлен скриптом
\texttt{analysis/scan\_spectral\_lab\_1d\_resolution.py}, который для
гармонического потенциала на отрезке $[-5,5]$ при разных $N$ сравнивает
$F_{\mathrm{levels}}^{(\mathrm{spec})}$ и
$F_{\mathrm{levels}}^{(\mathrm{FDM})}$ и записывает численные значения
и относительную ошибку в файлы
\texttt{results/spectral\_lab\_1d\_resolution.txt} и
\texttt{results/spectral\_lab\_1d\_resolution.csv}.

\paragraph{Постановка.}

\begin{itemize}
  \item Выбирается один фиксированный потенциал (например, гармонический
        осциллятор) с разумными параметрами $m$, $k$.
  \item Рассматривается набор размеров решётки $N \in \{100, 200, 400,
        800, \ldots\}$ при фиксированном интервале по $x$.
  \item Для каждой решётки строится оператор $H$, находится спектр и
        вычисляется $F_{\mathrm{levels}}^{(\mathrm{spec})}$ с выбранной
        весовой функцией $w(E)$.
  \item По тем же данным считается FDM-прокси
        $F_{\mathrm{levels}}^{(\mathrm{FDM})}$ и относительная ошибка
        \[
          \varepsilon(N) =
          \frac{\bigl|F_{\mathrm{levels}}^{(\mathrm{spec})}
          - F_{\mathrm{levels}}^{(\mathrm{FDM})}\bigr|}
               {\bigl|F_{\mathrm{levels}}^{(\mathrm{spec})}\bigr|}.
        \]
\end{itemize}

\paragraph{Ожидаемый результат.}

Ожидается, что при увеличении $N$ ошибка $\varepsilon(N)$ будет
падать или по крайней мере стабилизироваться на приемлемом уровне
для разумных значений $N$. Этот эксперимент задаёт \emph{оперативный
диапазон} размерностей решётки, в котором FDM-прокси может считаться
достаточно точным для R\&D-целей.

\paragraph{Фактический исход (Spectral Lab v1).}

Численный запуск эксперимента SL-1 для гармонического осциллятора
показал, что спектральный функционал уровней
$F_{\mathrm{levels}}^{(\mathrm{spec})}$ стабилизируется около значения
порядка $1.25$ при увеличении числа узлов решётки $N$, тогда как
FDM-прокси $F_{\mathrm{levels}}^{(\mathrm{FDM})}$ остаётся на уровне
порядка $0.30$. Относительная ошибка
$\varepsilon(N) \approx 0.75$ практически не меняется для
$N \in \{100, 200, 400, 800\}$. Это означает, что текущая форма
FDM-функционала для $F_{\mathrm{levels}}$ в Spectral Lab v1 даёт
существенно смещённую оценку и требует пересмотра.

\paragraph{SL-1b: линейная калибровка FDM-прокси.}

Для уменьшения систематического смещения между
$F_{\mathrm{levels}}^{(\mathrm{spec})}$ и
$F_{\mathrm{levels}}^{(\mathrm{FDM})}$ в Spectral Lab v1 введена
линейная FDM-модель
\[
  F_{\mathrm{levels}}^{(\mathrm{FDM,lin})}
  \;=\;
  a \, F_{\mathrm{naive}} + b,
\]
где $F_{\mathrm{naive}}$ --- прежняя FDM-аппроксимация по потенциалу
$V(x)$. Калибровочный скрипт
\texttt{analysis/calibrate\_f\_levels\_fdm\_1d.py} подбирает
коэффициенты $(a,b)$ методом наименьших квадратов по данным SL-1
(гармонический потенциал, несколько значений $N$). В текущих
расчётах получается почти постоянная поправка: $a$ мало по модулю,
$b \approx 1.2557$, а среднеквадратичная ошибка относительно
спектрального $F_{\mathrm{levels}}$ падает до $\sim 6 \cdot 10^{-4}$
на калибровочном наборе. Это подтверждает, что исходный
$F_{\mathrm{naive}}$ содержит мало информативной структуры и требует
более содержательного переопределения.

\subsubsection{Эксперимент SL-2: игрушечная спектральная ``жёсткость''
$\chi_{\mathrm{spec}}^{(1D)}$}

Второй эксперимент направлен на поиск одномерного аналога
спектральной ``жёсткости'' или электроотрицательности
$\chi_{\mathrm{spec}}$. Идея состоит в том, чтобы изучить семейство
потенциалов $V(x; \lambda)$ с параметром жёсткости $\lambda$ и
посмотреть, можно ли определить функционал
$\chi_{\mathrm{spec}}^{(1D)}(\lambda)$, монотонно отражающий
изменения жёсткости.

\paragraph{Постановка.}

\begin{itemize}
  \item Выбирается параметризованное семейство потенциалов
        $V(x;\lambda)$ (например, гармонический осциллятор с
        разными $k$ или двойная яма с меняющейся высотой/шириной
        барьера).
  \item Для каждого значения $\lambda$ строится спектр оператора
        $H(\lambda)$ и вычисляется $F_{\mathrm{levels}}^{(\mathrm{spec})}
        (\lambda)$ с фиксированной весовой функцией $w(E)$.
  \item Определяется игрушечный показатель
        \[
          \chi_{\mathrm{spec}}^{(1D)}(\lambda)
          = \frac{F_{\mathrm{levels}}^{(\mathrm{spec})}(\lambda)}
                 {\mathcal{N}(\lambda)},
        \]
        где $\mathcal{N}(\lambda)$ --- выбранная нормировка
        (например, число уровней в определённом энергетическом окне
        или простая функция параметров потенциала).
  \item Анализируется монотонность и стабильность
        $\chi_{\mathrm{spec}}^{(1D)}(\lambda)$ при изменении $\lambda$.
\end{itemize}

\paragraph{Ожидаемый результат.}

Если удаётся подобрать такую нормировку $\mathcal{N}(\lambda)$ и
вес $w(E)$, при которых $\chi_{\mathrm{spec}}^{(1D)}(\lambda)$ ведёт
себя монотонно и устойчиво при изменении жёсткости потенциала, это
даёт численный прототип для будущих определений спектральной
электроотрицательности в более реалистичных конфигурациях.

\paragraph{Выбор прототипов $\chi_{\mathrm{spec}}^{(1D)}$ и спектральной мягкости.}

Численные результаты для семейства гармонических потенциалов
$V(x;\lambda) = \tfrac{1}{2}\lambda x^2$ показывают, что величины
$\chi_0(\lambda)$, $\chi_{\mathrm{avg},5}(\lambda)$,
$\chi_{\mathrm{avg},10}(\lambda)$ и энергетический хвост
$\chi_{\mathrm{tail}}(\lambda)$ монотонно растут при увеличении
$\lambda$, тогда как спектральный функционал уровней
$F_{\mathrm{levels}}^{(\mathrm{spec})}(\lambda)$ с фиксированной
гауссовой весовой функцией убывает.

В качестве игрушечного прототипа спектральной ``жёсткости''
в одномерной модели мы фиксируем
\[
  \chi_{\mathrm{spec}}^{(1D)}(\lambda)
  \;\equiv\;
  \chi_{\mathrm{avg},10}(\lambda)
  \;=\;
  \frac{1}{10}\sum_{k=0}^{9} E_k(\lambda),
\]
то есть среднюю энергию первых десяти уровней гамильтониана
$H(\lambda)$. Эта величина монотонно растёт с $\lambda$ и ведёт себя
как $\chi_{\mathrm{spec}}^{(1D)} \sim \sqrt{\lambda}$, что согласуется
с ожидаемой зависимостью спектра гармонического осциллятора от
параметра жёсткости.

Одновременно функционал
\[
  s_{\mathrm{spec}}^{(1D)}(\lambda)
  \;\equiv\;
  F_{\mathrm{levels}}^{(\mathrm{spec})}(\lambda; w(E) = e^{-E^2/2}),
\]
который убывает при увеличении $\lambda$, естественно интерпретировать
как спектральную ``мягкость'' одномерного потенциала. Таким образом,
в Spectral Lab v1 появляется пара сопряжённых характеристик:
спектральная жёсткость $\chi_{\mathrm{spec}}^{(1D)}$ и спектральная
мягкость $s_{\mathrm{spec}}^{(1D)}$, которые качественно отражают
поведение спектра при изменении параметра жёсткости $\lambda$.

\subsection{Дальнейшая дорожная карта тома II}

В следующих разделах предполагается детализировать и реализовать:

\begin{itemize}
  \item расширение Spectral Lab до двумерных операторов и простых
        ``атомоподобных'' потенциалов;
  \item постепенный отход d-блока от чисто эмпирической калибровки
        по Паулингу к собственным геометрическим и спектральным
        моделям;
  \item генератор молекулярных графов с циклами и интеграцию
        crossing-слоя сложности;
  \item уточнение и углубление ядерного модуля и спектрального
        моста между геометрической таблицей и $F_{\mathrm{nuc}}$.
\end{itemize}

\subsection{Geom–nuclear–FDM: первый связанный стенд}

На уровне версий QSG v5.x/v6.0 собран комбинированный слой
geom--nuclear--complexity: таблица
\texttt{geom\_nuclear\_complexity\_summary.csv}, объединяющая
геометрический атом (роли, $\chi_{\mathrm{spec}}$, индексы $D/A$),
FDM-комплексность (средние по росту деревьев) и ядерные индикаторы.
На этом слое запущен первый анализ корреляций между
нормированной FDM-комплексностью и ядерными характеристиками
(\texttt{analysis/analyze\_fdm\_vs\_nucleus.py}, отчёт
\texttt{results/fdm\_vs\_nucleus\_stats.txt}).

После исправления поиска ядерных CSV комбинированная таблица содержит
поля `band_width`, $N_{\mathrm{best}}$, $N_{\mathrm{best}}/Z$ и
нейтронный избыток для большей части элементов H--Kr. Скрипт
`analyze_fdm_vs_nucleus` даёт первые ненулевые Spearman-корреляции
между нормированной FDM-комплексностью и ядерными параметрами (см.
`docs/05_decision_log.md`). Эти связи пока умеренные и нестабильные,
поэтому geom--nuclear--FDM фронт остаётся в статусе R\&D; дальнейшее
развитие требует более полного и однородного ядерного датасета.
