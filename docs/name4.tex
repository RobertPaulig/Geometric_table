\documentclass[12pt,a4paper]{article}
\usepackage{fontspec}
\usepackage{polyglossia}
\setmainlanguage{russian}
\setotherlanguage{english}
\setmainfont{Times New Roman}
\setsansfont{Arial}
\setmonofont{Consolas}
\usepackage{hyperref}
\usepackage{geometry}
\geometry{margin=2.5cm}
\begin{document}
% name4.tex — Том II: развитие модели QSG v6.x+

\section*{Том II. Развитие модели QSG: версии v6.x и далее}

Настоящий текст продолжает изложение, начатое в документе
\emph{name3.tex}, где фиксированы онтология спектральной среды,
геометрическая таблица, версия QSG v5.0 и введён минимальный
Spectral Lab v1. Здесь собираются рабочие записки и расширения,
относящиеся к будущим версиям модели (условно QSG v6.x и далее),
включая углублённый Spectral Lab, генераторы с циклами, d-блок,
ядерный мост и дальнейшее развитие спектральной гравитации.

\subsection{Обзор направления QSG v6.x}

В качестве отправной точки принимается следующая конфигурация:

\begin{itemize}
  \item \textbf{Зелёная зона QSG v5.0:} геометрический атом H--Kr,
        D/A-индексы и роли, деревьевой Grower, функционал сложности
        (v1 и FDM) с константами $\alpha_{\mathrm{tree}}$, а также
        мягкость $s(Z)$ для ростовой динамики.
  \item \textbf{Spectral Lab v1 (QSG v6.0 в зачатке):} одномерный
        оператор $\hat{H}$ на решётке, спектр, DOS/LDOS и игрушечный
        функционал $F_{\mathrm{levels}}$ с FDM-прокси.
  \item \textbf{Жёлтая зона:} континуальный оператор $\hat{H}[\theta]$,
        честный DOS/LDOS, crossing-слой сложности, полноценный d-блок,
        строгий мост geom $\leftrightarrow F_{\mathrm{nuc}}$.
  \item \textbf{Красная зона:} спектральная гравитация и космология,
        многоуровневая солитонная онтология и полная интеграция всех
        слоёв в единую спектральную теорию.
\end{itemize}

 \subsection{HETERO-1A как прикладной контур QSG}

 Линия HETERO-1A (acid / hetero growth) выросла из разделов \emph{name3} о геометрических функционалах и служит прикладной проверкой идей QSG. Здесь мы переносим абстрактные постулаты в конкретные скрипты (\texttt{analysis/chem/*}) и следим, чтобы экспериментальные метрики оставались совместимыми с $F_{\mathrm{levels}}$ и $F_{\mathrm{geom}}$:
 \begin{itemize}
   \item \textbf{Связь со спектральной гравитацией.} Пары классов (alcohol/ether, первичный/вторичный амин и т.д.) трактуются как локальные вариации спектра Среды. Любая метрика (ROC-AUC, collision rates) описывает, насколько две конфигурации различимы в духе \emph{name3}.
   \item \textbf{Tie-aware AUC.} Введён взвешенный Mann--Whitney AUC с агрегированием по блокам равных энергий. Это реализует требование \emph{name3} об инвариантности к перестановкам внутри спектрально неразличимых состояний и устраняет зависимость от порядка.
   \item \textbf{Динамический neg-control gate.} Вместо фиксированной оценки 0.60 мы берём точную комбинаторную квантиль $q_{0.95}^{\text{null}}(m,n)$ и добавляем регулируемый запас (по умолчанию 0.06). Такой гейт интерпретируется как “спектральный запрет” и логируется (\texttt{fp\_neg\_auc\_gate}, \texttt{fp\_neg\_auc\_null\_q}, \texttt{fp\_neg\_auc\_slack}).
   \item \textbf{Slack и Grower/FDM.} Колонка \texttt{fp\_neg\_auc\_slack} показывает, насколько далеко мы от порога. Если slack $\approx 0$, это сигнал к пересмотру параметров grower/FDM или увеличению margin, чтобы не нарушить спектральные ограничители \emph{name3}.
 \end{itemize}

 Таким образом, HETERO-1A выступает “витриной” томов \emph{name3/name4}: мы валидируем спектральные постулаты на конкретных гетеро-семействах, логируем пороги/квантили и используем ту же терминологию (инварианты, функционалы, slack), что и в теоретической части.

Задача тома II --- пошагово продвигать модель от текущего состояния
к более полному спектральному описанию, при этом каждый шаг должен
быть \emph{привязан к коду и численным экспериментам}, а не оставаться
на уровне чисто словесных деклараций.

\subsection{Spectral Lab v1 как R\&D-площадка}

Цель Spectral Lab v1 --- служить минимальной лабораторией для проверки
идей о спектральном операторе $\hat{H}$ и функционале уровней
$F_{\mathrm{levels}}$, а также для отработки связи между спектральной
картиной и FDM-приближениями.

В \emph{name3.tex} описан базовый одномерный оператор
\[
  H = -\frac{1}{2m}\,\frac{\mathrm{d}^2}{\mathrm{d}x^2} + V(x)
\]
на равномерной решётке, реализованный в модуле
\texttt{core/spectral\_lab\_1d.py}, и две версии функционала уровней:
спектральную $F_{\mathrm{levels}}^{(\mathrm{spec})}$ и FDM-прокси
$F_{\mathrm{levels}}^{(\mathrm{FDM})}$ (модуль
\texttt{core/f\_levels\_1d.py}). Здесь мы фиксируем план экспериментов,
которые должны сделать Spectral Lab содержательным шагом в сторону
настоящего $\hat{H}[\theta]$.

\subsubsection{Эксперимент SL-1: разрешение решётки и согласие spec/FDM}

Первая задача Spectral Lab v1 --- количественно проверить, насколько
FDM-прокси для $F_{\mathrm{levels}}$ приближают спектральные значения
при разумных параметрах. Для этого вводится эксперимент SL-1.

В реализации QSG v6.0 эксперимент SL-1 оформлен скриптом
\texttt{analysis/scan\_spectral\_lab\_1d\_resolution.py}, который для
гармонического потенциала на отрезке $[-5,5]$ при разных $N$ сравнивает
$F_{\mathrm{levels}}^{(\mathrm{spec})}$ и
$F_{\mathrm{levels}}^{(\mathrm{FDM})}$ и записывает численные значения
и относительную ошибку в файлы
\texttt{results/spectral\_lab\_1d\_resolution.txt} и
\texttt{results/spectral\_lab\_1d\_resolution.csv}.

\paragraph{Постановка.}

\begin{itemize}
  \item Выбирается один фиксированный потенциал (например, гармонический
        осциллятор) с разумными параметрами $m$, $k$.
  \item Рассматривается набор размеров решётки $N \in \{100, 200, 400,
        800, \ldots\}$ при фиксированном интервале по $x$.
  \item Для каждой решётки строится оператор $H$, находится спектр и
        вычисляется $F_{\mathrm{levels}}^{(\mathrm{spec})}$ с выбранной
        весовой функцией $w(E)$.
  \item По тем же данным считается FDM-прокси
        $F_{\mathrm{levels}}^{(\mathrm{FDM})}$ и относительная ошибка
        \[
          \varepsilon(N) =
          \frac{\bigl|F_{\mathrm{levels}}^{(\mathrm{spec})}
          - F_{\mathrm{levels}}^{(\mathrm{FDM})}\bigr|}
               {\bigl|F_{\mathrm{levels}}^{(\mathrm{spec})}\bigr|}.
        \]
\end{itemize}

\paragraph{Ожидаемый результат.}

Ожидается, что при увеличении $N$ ошибка $\varepsilon(N)$ будет
падать или по крайней мере стабилизироваться на приемлемом уровне
для разумных значений $N$. Этот эксперимент задаёт \emph{оперативный
диапазон} размерностей решётки, в котором FDM-прокси может считаться
достаточно точным для R\&D-целей.

\paragraph{Фактический исход (Spectral Lab v1).}

Численный запуск эксперимента SL-1 для гармонического осциллятора
показал, что спектральный функционал уровней
$F_{\mathrm{levels}}^{(\mathrm{spec})}$ стабилизируется около значения
порядка $1.25$ при увеличении числа узлов решётки $N$, тогда как
FDM-прокси $F_{\mathrm{levels}}^{(\mathrm{FDM})}$ остаётся на уровне
порядка $0.30$. Относительная ошибка
$\varepsilon(N) \approx 0.75$ практически не меняется для
$N \in \{100, 200, 400, 800\}$. Это означает, что текущая форма
FDM-функционала для $F_{\mathrm{levels}}$ в Spectral Lab v1 даёт
существенно смещённую оценку и требует пересмотра.

\paragraph{SL-1b: линейная калибровка FDM-прокси.}

Для уменьшения систематического смещения между
$F_{\mathrm{levels}}^{(\mathrm{spec})}$ и
$F_{\mathrm{levels}}^{(\mathrm{FDM})}$ в Spectral Lab v1 введена
линейная FDM-модель
\[
  F_{\mathrm{levels}}^{(\mathrm{FDM,lin})}
  \;=\;
  a \, F_{\mathrm{naive}} + b,
\]
где $F_{\mathrm{naive}}$ --- прежняя FDM-аппроксимация по потенциалу
$V(x)$. Калибровочный скрипт
\texttt{analysis/calibrate\_f\_levels\_fdm\_1d.py} подбирает
коэффициенты $(a,b)$ методом наименьших квадратов по данным SL-1
(гармонический потенциал, несколько значений $N$). В текущих
расчётах получается почти постоянная поправка: $a$ мало по модулю,
$b \approx 1.2557$, а среднеквадратичная ошибка относительно
спектрального $F_{\mathrm{levels}}$ падает до $\sim 6 \cdot 10^{-4}$
на калибровочном наборе. Это подтверждает, что исходный
$F_{\mathrm{naive}}$ содержит мало информативной структуры и требует
более содержательного переопределения.

\subsubsection{Эксперимент SL-2: игрушечная спектральная ``жёсткость''
$\chi_{\mathrm{spec}}^{(1D)}$}

Второй эксперимент направлен на поиск одномерного аналога
спектральной ``жёсткости'' или электроотрицательности
$\chi_{\mathrm{spec}}$. Идея состоит в том, чтобы изучить семейство
потенциалов $V(x; \lambda)$ с параметром жёсткости $\lambda$ и
посмотреть, можно ли определить функционал
$\chi_{\mathrm{spec}}^{(1D)}(\lambda)$, монотонно отражающий
изменения жёсткости.

\paragraph{Постановка.}

\begin{itemize}
  \item Выбирается параметризованное семейство потенциалов
        $V(x;\lambda)$ (например, гармонический осциллятор с
        разными $k$ или двойная яма с меняющейся высотой/шириной
        барьера).
  \item Для каждого значения $\lambda$ строится спектр оператора
        $H(\lambda)$ и вычисляется $F_{\mathrm{levels}}^{(\mathrm{spec})}
        (\lambda)$ с фиксированной весовой функцией $w(E)$.
  \item Определяется игрушечный показатель
        \[
          \chi_{\mathrm{spec}}^{(1D)}(\lambda)
          = \frac{F_{\mathrm{levels}}^{(\mathrm{spec})}(\lambda)}
                 {\mathcal{N}(\lambda)},
        \]
        где $\mathcal{N}(\lambda)$ --- выбранная нормировка
        (например, число уровней в определённом энергетическом окне
        или простая функция параметров потенциала).
  \item Анализируется монотонность и стабильность
        $\chi_{\mathrm{spec}}^{(1D)}(\lambda)$ при изменении $\lambda$.
\end{itemize}

\paragraph{Ожидаемый результат.}

Если удаётся подобрать такую нормировку $\mathcal{N}(\lambda)$ и
вес $w(E)$, при которых $\chi_{\mathrm{spec}}^{(1D)}(\lambda)$ ведёт
себя монотонно и устойчиво при изменении жёсткости потенциала, это
даёт численный прототип для будущих определений спектральной
электроотрицательности в более реалистичных конфигурациях.

\paragraph{Выбор прототипов $\chi_{\mathrm{spec}}^{(1D)}$ и спектральной мягкости.}

Численные результаты для семейства гармонических потенциалов
$V(x;\lambda) = \tfrac{1}{2}\lambda x^2$ показывают, что величины
$\chi_0(\lambda)$, $\chi_{\mathrm{avg},5}(\lambda)$,
$\chi_{\mathrm{avg},10}(\lambda)$ и энергетический хвост
$\chi_{\mathrm{tail}}(\lambda)$ монотонно растут при увеличении
$\lambda$, тогда как спектральный функционал уровней
$F_{\mathrm{levels}}^{(\mathrm{spec})}(\lambda)$ с фиксированной
гауссовой весовой функцией убывает.

В качестве игрушечного прототипа спектральной ``жёсткости''
в одномерной модели мы фиксируем
\[
  \chi_{\mathrm{spec}}^{(1D)}(\lambda)
  \;\equiv\;
  \chi_{\mathrm{avg},10}(\lambda)
  \;=\;
  \frac{1}{10}\sum_{k=0}^{9} E_k(\lambda),
\]
то есть среднюю энергию первых десяти уровней гамильтониана
$H(\lambda)$. Эта величина монотонно растёт с $\lambda$ и ведёт себя
как $\chi_{\mathrm{spec}}^{(1D)} \sim \sqrt{\lambda}$, что согласуется
с ожидаемой зависимостью спектра гармонического осциллятора от
параметра жёсткости.

Одновременно функционал
\[
  s_{\mathrm{spec}}^{(1D)}(\lambda)
  \;\equiv\;
  F_{\mathrm{levels}}^{(\mathrm{spec})}(\lambda; w(E) = e^{-E^2/2}),
\]
который убывает при увеличении $\lambda$, естественно интерпретировать
как спектральную ``мягкость'' одномерного потенциала. Таким образом,
в Spectral Lab v1 появляется пара сопряжённых характеристик:
спектральная жёсткость $\chi_{\mathrm{spec}}^{(1D)}$ и спектральная
мягкость $s_{\mathrm{spec}}^{(1D)}$, которые качественно отражают
поведение спектра при изменении параметра жёсткости $\lambda$.

\subsection{Дальнейшая дорожная карта тома II}

В следующих разделах предполагается детализировать и реализовать:

\begin{itemize}
  \item расширение Spectral Lab до двумерных операторов и простых
        ``атомоподобных'' потенциалов;
  \item постепенный отход d-блока от чисто эмпирической калибровки
        по Паулингу к собственным геометрическим и спектральным
        моделям;
  \item генератор молекулярных графов с циклами и интеграцию
        crossing-слоя сложности;
  \item уточнение и углубление ядерного модуля и спектрального
        моста между геометрической таблицей и $F_{\mathrm{nuc}}$.
  \item равновесный химический контур для алканов (tree-only): fixed-$N$ MCMC, диагностика mixing, строгая coverage и продуктовый pipeline C15/C16.
\end{itemize}

\subsection{Пакет CHEM-VALIDATION: равновесные алканы (C4--C16)}

В рамках химического контура QSG реализован и закрыт (как продуктовый стенд) контур
валидации на классе \emph{tree-only} алканов: связные деревья на $N$ вершинах (углероды),
$c=N-1$ рёбер, цикломатическое число $0$, ограничение валентности $\max\deg\le 4$.
Цель --- получать воспроизводимую равновесную меру по топологиям при фиксированном $N$
и использовать её как эталон для ранжирования изомеров по ``энергии сложности''
(Mode~A: FDM-only на деревьях).

\paragraph{Топологии, пометки и дегенерация.}
Рассматриваются \emph{помеченные} деревья (labeled, вершины $1..N$) и
индуцированные ими \emph{непомеченные} топологии (unlabeled). При переходе к топологиям
учитывается комбинаторная дегенерация
\[
  g(\tau)=\frac{N!}{|\mathrm{Aut}(\tau)|},
\]
где $\mathrm{Aut}(\tau)$ --- группа автоморфизмов топологии $\tau$.
Это обязательный множитель при сопоставлении ``энергий топологии'' с частотами.

\paragraph{Рост $\neq$ равновесие.}
Частоты, наблюдаемые в grower-процессе, отражают прежде всего \emph{proposal/kernel bias}
и не являются равновесием при фиксированном $N$.
Равновесная мера $P_{\mathrm{eq}}$ строится отдельным слоем:
fixed-$N$ MCMC по пространству помеченных деревьев
(leaf-rewire + поправка Гастингса). Для малых $N$ используется exact-база (Pr\"ufer),
что позволяет проверять корректность MCMC ``по истине''.

\paragraph{Инвариантность энергии на деревьях.}
Критический инженерный инвариант: энергия (FDM) на tree-only должна быть
\emph{перестановочно-инвариантной}. Для этого перед вычислением FDM выполняется
канонизация дерева (детерминированное переименование вершин), после чего для одной
и той же топологии дисперсия энергии по различным пометкам стремится к нулю.
Этот шаг восстановил согласованность $P_{\mathrm{pred}}$ и exact-базы на малых $N$
и устранил ложные расхождения, вызванные label-dependence.

\paragraph{DoD-метрики mixing и coverage.}
Для больших $N$ (C15/C16) вместо ``стрельбы вслепую'' бюджет шагов калибруется по
guardrail-метрикам (несколько стартов, несколько цепей):
\begin{itemize}
  \item $KL_{\mathrm{max\_pairwise}}\le 0.01$ (макс. KL между цепями),
  \item $KL_{\mathrm{split\_max}}\le 0.01$ (KL между первой/второй половинами),
  \item $\widehat{R}_{\mathrm{energy}}\le 1.05$,
  \item $ESS_{\mathrm{energy}}\ge 500$,
  \item \textbf{строгая coverage}: $n_{\mathrm{unique}}$ совпадает с ожидаемым числом
        непомеченных топологий (C15: 4347; C16: 10359).
\end{itemize}
В CHEM-VALIDATION-5 бюджет задаётся как \emph{total steps} по всем
$\#\mathrm{starts}\times\#\mathrm{chains}$; число шагов на цепь:
$\mathrm{steps\_per\_chain}=\mathrm{steps\_total}/(\#\mathrm{starts}\cdot\#\mathrm{chains})$.
Для финальных прогонов фиксируются также \texttt{thin=10} и \texttt{burnin\_frac=0.30}.

\paragraph{Distributed-контур (EQ-DIST-1).}
Для устойчивости к тайм-лимитам и масштабирования на произвольное железо введены
\emph{work units} (аналог GIMPS): генератор задач (make\_tasks), воркер (worker) и
агрегатор (aggregate) с режимом \texttt{--status} (READY/MISSING) и опциональной
двойной верификацией на выбранных шагах. Энергия переиспользуется через
персистентный energy-cache, а агрегаты фиксируются commit+push.

\paragraph{Финальный результат C15/C16 (Mode A, fixed budget).}
Контур CHEM-VALIDATION-5 закрыт на строгих DoD-порогах и 100\% coverage:

\begin{center}
\begin{tabular}{l r r}
\hline
Метрика & C15 (финал) & C16 (финал) \\
\hline
$\mathrm{steps\_total}$ & $1.60\times 10^{8}$ & $3.20\times 10^{8}$ \\
$\mathrm{steps\_per\_chain}$ & $1.6\times 10^{7}$ & $3.2\times 10^{7}$ \\
$KL_{\mathrm{max\_pairwise}}$ & $0.004114$ & $0.004858$ \\
$KL_{\mathrm{split\_max}}$ & $0.008336$ & $0.009774$ \\
$\widehat{R}_{\mathrm{energy}}$ (max) & $1.000000$ & $1.000001$ \\
$ESS_{\mathrm{energy}}$ (min) & $3.79\times 10^{5}$ & $6.73\times 10^{5}$ \\
Coverage ($n_{\mathrm{unique}}$) & $4347/4347$ & $10359/10359$ \\
\hline
\end{tabular}
\end{center}

\paragraph{Комментарий о ``статистическом полу'' $KL_{\mathrm{split}}$.}
На больших пространствах (C16: $K\approx 10^{4}$ топологий) $KL_{\mathrm{split}}$
приближается к нижнему уровню, обусловленному конечной длиной цепей и дискретностью
эмпирических частот. Практически это означает: дальнейшее увеличение бюджета даёт
уменьшение $KL_{\mathrm{max\_pairwise}}$ быстрее, чем $KL_{\mathrm{split}}$.
Инженерный порог $0.01$ остаётся разумным gate, но требует интерпретации вместе с
coverage и $\widehat{R}/ESS$.

\paragraph{Следующий инженерный шаг.}
Далее контур упаковывается в стабильный продуктовый pipeline:
единые CLI/конфиги, детерминированные seeds, единый формат отчётов, регрессионные тесты
на малых $N$ (exact vs MCMC) и расширение за пределы tree-only (циклы/конформации)
с сохранением раздельной трактовки proposal bias и равновесия.


\subsection{Geom–nuclear–FDM: первый связанный стенд}

На уровне версий QSG v5.x/v6.0 собран комбинированный слой
geom--nuclear--complexity: таблица
\texttt{geom\_nuclear\_complexity\_summary.csv}, объединяющая
геометрический атом (роли, $\chi_{\mathrm{spec}}$, индексы $D/A$),
FDM-комплексность (средние по росту деревьев) и ядерные индикаторы.
Комбинированная таблица формируется скриптом
\texttt{analysis/analyze\_geom\_nuclear\_complexity.py}, который
объединяет геометрические индексы, FDM-комплексность и ядерные
таблицы (\texttt{geom\_isotope\_bands.csv},
\texttt{geom\_nuclear\_map.csv}), а также вычисляет производные
поля (`band\_width`, $N_{\mathrm{best}}/Z$, нейтронный избыток).
На этом слое запущен первый анализ корреляций между нормированной
FDM-комплексностью и ядерными характеристиками
(\texttt{analysis/analyze\_fdm\_vs\_nucleus.py}, отчёт
\texttt{results/fdm\_vs\_nucleus\_stats.txt}).

Для элементов H--Xe глобальные Spearman-корреляции показывают
умеренные связи между нормированной FDM-сложностью и ядерными
параметрами (шириной полосы `band\_width`, отношением
`Nbest\_over\_Z` и нейтронным избытком), а по ролям и отдельным
группам (terminator/bridge/hub, d-блок, living hubs) картина
остаётся шумной из-за малых выборок. Этот слой зафиксирован
как R\&D-стенд: он демонстрирует наличие слабых/умеренных связей
между геометрической сложностью и ядерными характеристиками, но
для строгих утверждений требуется более полный и однородный
ядерный датасет.

\subsection{Пакет CY-1: первый стенд циклов и crossing-proxy (QSG v6.x)}

В рамках пакета CY-1 для QSG v6.x введён первый стенд циклов:
отдельный скрипт \texttt{analysis/scan\_cycles\_vs\_params.py}
исследует область параметров роста, в которой стохастический Grower
начинает порождать графы с ненулевым цикломатическим числом.
На этом уровне вводится простой crossing-proxy (циклонaгрузка на
вершину), который служит тренировочной площадкой для будущего
топологического слоя сложности (crossing-number и родственные
инварианты).

\paragraph{Режимы loopy-growth (CY-1-A/B).}
После подтверждения деревьевого предела QSG v5.0 в пакете CY-1
введён R\&D-режим роста с циклами (loopy-growth). На уровне ядра
расширены параметры роста \texttt{GrowthParams} флагом
\texttt{allow\_cycles} и двумя числовыми параметрами
(\texttt{max\_extra\_bonds}, \texttt{p\_extra\_bond}), которые по
умолчанию выключены. Дополнительный слой \emph{loopy overlay}
добавляет случайные связи между уже существующими вершинами поверх
готового дерева и тем самым порождает графы с ненулевым
цикломатическим числом.

На основе скана параметров роста (\texttt{results/cycle\_param\_scan.csv})
выбраны эталонные режимы CY-1-A (умеренная доля циклов) и
CY-1-B (агрессивный циклонасыщенный рост). Скрипт
\texttt{analysis/analyze\_loopy\_modes.py} измеряет для них долю
графов с циклами, распределение цикломатического числа и среднюю
циклонагрузку на вершину, что служит первым численным прототипом
crossing-слоя топологической сложности.
Слой FDM-сложности при этом расширен R\&D-вариантом
\texttt{fdm\_loopy}: для заданной FDM-комплексности
$C_{\mathrm{FDM}}(G)$ вводится штраф
\\[0.3em]
\hspace*{2em}$
  C_{\mathrm{FDM}}^{(\mathrm{loopy})}(G)
  \;=\;
  C_{\mathrm{FDM}}(G)\,\Bigl(
    1 + \alpha_{\mathrm{cycle}}\,\mu(G)
      + \alpha_{\mathrm{load}}\,\tfrac{\mu(G)}{n(G)}
  \Bigr),
$\\[0.3em]
где $\mu(G)$ --- цикломатическое число, $n(G)$ --- число вершин.
Для деревьев ($\mu(G)=0$) закон совпадает с базовым FDM-режимом,
а для графов CY-1-A/B дополнительный множитель делает циклы
«дорогими» с точки зрения FDM-закона сложности.
Дополнительно для малых графов (обычно $n\leq 8$) вводится
игрушечный crossing-number $cr_{\mathrm{circle}}(G)$ в модели
``вершины на окружности, рёбра --- хорды'', реализованный в модуле
\texttt{core/crossing.py}. Анализ по режимам CY-1-A/B показывает,
что плотность пересечений $cr_{\mathrm{circle}}(G)$ разумно
коррелирует как с циклонагрузкой $\mu(G)/n(G)$, так и с фактором
штрафа в FDM-законе, что служит первым численным подтверждением
согласованности crossing-aware FDM-слоя с дискретным прототипом
пересечений. Для дальнейшего R\&D введён расширенный backend
\texttt{fdm\_loopy\_cross}, в котором поверх циклового штрафа
используется дополнительный множитель
$(1 + \beta_{\mathrm{cross}} \cdot \mathrm{crossing\_proxy}(G))$,
где при малых размерах графа proxy строится по
$cr_{\mathrm{circle}}(G)$, а при больших используется циклонагрузка.
Параметр $\beta_{\mathrm{cross}}$ калибруется отдельно и не влияет
на базовый FDM-режим v5.0.

\paragraph{Пакет TEMP-1: температура среды и шум роста.}

В ростовых параметрах QSG v6.x введён явный параметр температуры $T$,
который деформирует эффективную вероятность продолжения ветви
$p_{\mathrm{continue}}$ так, что при $T=1$ воспроизводится базовый
деревьевой режим QSG v5.0, при $T<1$ деревья и loopy-графы в среднем
растут длиннее, а при $T>1$ --- короче и легче распадаются.
Конфигурации среды \texttt{growth\_env\_cold.yaml} и
\texttt{growth\_env\_hot.yaml} задают первые "холодные" и "горячие"
режимы для loopy-роста, а стенд
\texttt{analysis/scan\_temperature\_effects.py} количественно описывает
зависимость среднего размера и циклонагрузки от $T$ для элементов
C, Si, O, S. Этот слой интерпретируется как первый технический шаг к
модели манаховского "шума среды" и спектральной мягкости/жёсткости
связей.

\end{document}
